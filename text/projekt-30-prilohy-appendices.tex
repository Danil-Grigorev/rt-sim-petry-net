% Tento soubor nahraďte vlastním souborem s přílohami (nadpisy níže jsou pouze pro příklad)
% This file should be replaced with your file with an appendices (headings below are examples only)

% Umístění obsahu paměťového média do příloh je vhodné konzultovat s vedoucím
% Placing of table of contents of the memory media here should be consulted with a supervisor
%\chapter{Obsah přiloženého paměťového média}

\chapter{Instalační manuál}
\label{chap:instal-tutor}

\section{Instalace simulátoru}
Pokud byste potřebovali odzkoušet simulační nástroj z této práci, tak bych doporučoval provést této kroky instalace.
\begin{enumerate}
    \item Nainstalovat si balíčky \texttt{python3.6}, \texttt{graphviz}, \texttt{python3-pip}, \texttt{git}.
    \item Nainstalovat knihovnu SNAKES podle nějakého z postupu v \href{https://www.ibisc.univ-evry.fr/~fpommereau/SNAKES/first-steps-with-snakes.html}{návodu}. \label{pip-snakes}
    Doporučený postup, nainstalovat to přes pip:

    \texttt{pip3 install git+git://github.com/fpom/snakes \-\-user}.
    \item Nainstalovat této knihovny přes \texttt{pip3 install xxx \-\-user} pro podporu MQTT, kde \texttt{xxx} je:
    \begin{itemize}
        \item \texttt{pahomqtt}
        \item \texttt{hbmqtt}
    \end{itemize}
    \item Naklonovat git repositář simulační knihovny do libovolného místa na disku:

    \href{https://github.com/Danil-Grigorev/rt-sim-petry-net}{\texttt{https://github.com/Danil-Grigorev/rt-sim-petry-net}}.
    \item Provést potřebné úpravy pro rozšíření \uv{gv} v poslední verzí knihovny SNAKES, a nakopírovat jiné rozšíření do složky pro rozšíření. Tá se obecně po kroku \ref{pip-snakes} nachází v \texttt{$\sim$/.local/lib/python3.6/site-packages/snakes/plugins}, ale může se to lišit v závislosti od nastavení balíčku Python. Seznám rozšíření pro kopírovaní/nahrazení ze složky \texttt{rt-sim-petry-net/plugins} je následující:
    \begin{itemize}
        \item \texttt{prior\_pl.py}
        \item \texttt{prob\_pl.py}
        \item \texttt{sim\_pl.py}
        \item \texttt{timed\_pl.py}
        \item \texttt{gv.py}
    \end{itemize}

\end{enumerate}

\section{MQTT broker}

V mém \href{https://github.com/Danil-Grigorev/rt-sim-petry-net}{repositáři} se taky nachází konfigurační soubor pro případ, že byste neměli k dispozici MQTT broker. Je to soubor \href{https://github.com/Danil-Grigorev/rt-sim-petry-net/blob/master/hbmqtt.conf}{hbmqtt.conf}, a abyste takový server spustili na lokálním počítači, stačí k tomu příkaz \texttt{hbmqtt -c hbmqtt.conf}.

Pro instalaci tohoto nástroje, stačí spustit příkaz \texttt{pip3 install hbmqtt}.

\section{Šablona pro simulator}

Běžný způsob použítí simulátoru předpokládá využití třídy \texttt{PNSim}, ale pro zjednodušení jsem se rozhodl vytvořit \href{https://github.com/Danil-Grigorev/rt-sim-petry-net/blob/master/template.py}{šablonu} pro spuštění simulátoru. Použítí je snadné: \ref{code:template-example}.

\begin{python}
    # Import metod ze sablony (*\label{code:template-example}*)
    from template import *
    nets = [] # Seznam Petriho siti pro simulaci
    (*\ldots*) # Implementace Petriho siti a pridani instanci do seznamu
    execute_nets(nets, sim_id='server') # Spusteni simulacni instance pod nazvem 'server'
\end{python}

\section{Ukázka externí aplikace}
\label{sec:external-app-example}

Postup pro vytvoření takové aplikace se nachází v kapitole \ref{subsec:external-app}, výsledný kód z komentáři je součástí \href{https://github.com/Danil-Grigorev/rt-sim-petry-net/blob/master/temperature_logger.py}{tohoto} souboru.

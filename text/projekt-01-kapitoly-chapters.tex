%===============================================================================
% Autoři: Michal Bidlo, Bohuslav Křena, Jaroslav Dytrych, Petr Veigend a Adam Herout 2018
\chapter{Úvod}
\chapter{Prehled teorie}
\label{prehled}

V teto kapitole jsou popsany teoretické zaklady Petriho sítí. Taky tato kapitola se soustředí na popís několíka podtříd petrího site, zejmena na „vysokourovňové Petrího Sítě“ a „Časované Petrího Sítě“ a  včetně přehledu míst, kdě se Petrího sítě použivají, nebo hrají významnou roli. Taky bude zmineno o simulacni teorii, zamerene na simulaci Petriho siti v realnem case. Taky se popise pojem distribovanych systemu a ukaze se jejich vztah z Petriho siti.

\subsection*{Petriho site}
V současné době se Petrího site často použivají jako modelovací prostředek pro modelovaní chovaní systemu, za účelem pochopení jeho slabších stránek. Použití Petrího síti pro tento učel je velice vhodne, malo modelovacích prostředku dokáže popsat ne jenom system jako celek, ale i zjistit slabší stránky jeho implementace.
\subsubsection*{Definice}

Petrího síť ($PN$) se sestavuje ze ctrz prvku, $PN = \left(P, T, I, O\right)$ kde:
  \begin{itemize}
    \item $P = \left\{p_1, p_2, \cdots , p_n\right\}$ je konečná množína míst, $n >= 0$; \\
    \item $T = \left\{t_1, t_2, \cdots , t_m\right\}$ je konečná množína přechodů, $m >= 0$; \\
    \item $I = T \rightarrow P^\infty$ je vstupní funkce, propojující přechody s množínou vstupních míst; \\
    \item $O = P \rightarrow T^\infty$ je výstupní funkce, propojující výstupní místa s množínou přechodů; \\
  \end{itemize}
P $\vee$ T = $\varnothing$, P $\wedge$ T != $\varnothing$. 

Petrího síť může provest přechod 

\subsection*{Graf Petriho site}

\subsubsection{Typy Petriho site}

\paragraph*{Znackovane petriho site}
Znackovane petriho seti (ZPS) zavadi takovy pojem jako "znacka". "Znacka" je zakladni prvek v ZPS, umoznujici jeji vykonovani. Vykonovani se provadi provedenim prehodu, behem ktereho se vymazou znacky ze mnoziny vstupnich mist, a prida se odpovidajici pocet do vystupnich mist. Tato operace se muze vykonovat dokud nezbyde zadny povoleny prechod.

Prechod v PS je povolen, dokud kazde ze vstupnich mist propojenych s danym prechodem obsahuje alespon jednu znacku. Znacky, umoznujici provedeni prechodu se jmenuji "umoznujici znacky".

Pri vykonovani seti Petri vznikaji dve posloupnosti - posloupnost znacek a posloupnost prechodu, ktere byly provedeny. Teto dve posloupnosti zcela popisuji vykonani Petriho seti.
%[Petri net theory and model... par:2.4 p 18-19]

Diky moznosti vykonovat Petriho sit, vznika prilezitost vytvorzit simulacni nastroj, ktery je zalozen na zminenych zasadech. 

%\paragraph*{Casovane petriho site}
\paragraph*{Vysokourovnove petriho site}
\subsection*{Distribovany system}
Definice distribovaneho systemu zni:
Distribovany system je kolekce na sobe nezavislych pocitatcu, ktery konecny uzivatel vnima jako celek. 
% http://barbie.uta.edu/~jli/Resources/MapReduce&Hadoop/Distributed%20Systems%20Principles%20and%20Paradigms.pdf

Distribovane systemy museji pouzivat decentralizovane algoritmy, ktere pozaduji nasledujici: 1. Zadny prvek nesmi vedet informaci o celkovem stavu systemu. 2. Prvek uvazuje jenom na zaklade jeho stavu. 3. Chyba v zadnem z prvku ne ovlivni chovani algoritmu. 4. Neprovadi se synchronizace podle globalnich hodin. 

Z definice taky vypliva, ze distribovany system se obecne sestavuje s vice se opakujicich prvku, ktere funguji nezavisle na sobe a nemaji pristup ke sdilene pameti, neboli kazdy prvek ma svoi vlastni. Sdileni stavu ve pripade distribovaneho systemu se provadi pomoci zasilani zprav.

Tento koncept je velice pouzitelny pro modelovani. Prvky budou reprezentovany Petriho sitmi, provazane komunikacnimi kanaly. Jelikoz kazda petriho sit je prvek nezavisly na vnejsich podminkach, ale jen na soucasnem stavu mist a prechodu a taky jejich kombinaci, tak se da to prohlasit decentralizovanym algoritmem. Petriho sit si muzeme predstavit jako nezavisly blok, kdyzto pro kommunikaci mezi temito bloky se da pouzit protokol MQTT(odkaz), ktery diky svym vlastnostem dokaze obslouzit neomezene mnozstvi klientu najednou. Pripad implementace (Boiler net) nazorne ukazuje pouziti dane myslenky na praktice.
\subsection*{Simulace}
\subsection*{Diskretni simulace}
Diskretni simulace je zalozena ve svem zakladu na algoritmu znamem pod nazvem "NextEvent". Tento algoritmus se ridi nasledujicim pseudokodem:
nainicializuj simulaci, cas a planovac udalosti
dokud je naplanovana dalsi udalost
  vyjmi prvni udalost ze seznamu
  $if cas_udalosti >= T_END:$
    konec simulace
  nastav cas na cas udalosti
  proved popis chovani udalosti
\subsection*{Simulace v realnem case}
Simulace v realnem case se lisi of diskretni simulace v jedne vlastnosti. Beh takove simulace musi byt synchronizovan s relanim casem. Obecne posuv v casove rovine se provadi kvazidistantnim krokem, pricemz ten krok musi byt dostatecne maly, na to aby simulace stacila cist vstupni parametry. ??? % Real-Time Simulation of Electric Vehicles for Distribution System Operation Assessment, Characteristics of Real-Time Simulation

Simulace v realnem case upravena pro ucely tohoto projektu je zalozena na vyse zminenem algoritmu "Next Event" s jedinou zmenou, ktera pridava cekani na synchronizaci simulacniho casu s realnim nebo na prichod nove udalosti. Algoritmus v tom pripade vypada nasledujim zpusobem:
nainicializuj simulaci, cas a planovac udalosti
dokud je naplanovana udalost
  podivej se na cas dalsi udalosti 
  $if cas_udalosti >= T_END:$
    konec simulace
  pockej na cas udalosti nebo na prichod nove udalosti
  if nova udalost:
    continue
  proved popis chovani udalosti

Tento algoritmus nachazi svoje pouziti ve mnoha simulacnich pripadech. Bezny priklad pouziti - pocitacove hry, kde uzivatele zaujme herni prostredi, ktere simuluje realni svet, vcetne casove roviny. Simulace v realnim case se taky pouziva, kdy obycejne diskretni simulace nestaci a cas neni zanedbatelny. Uzivatel muze chtit sledovat system v prubehu simulace, za ucelem zjisteni anomalii v chovani jeho prvku nebo zvyseni kvality systemu pred jeho nasazenim do provozu. Takovy pristup se pouziva v simulace "Hardware-in-the-loop"(odkaz).
\subsection*{Hardware-in-the-loop(HWIL)}
\subsection*{Applikace}
\paragraph*{Obsah}

1. \url{https://www-tandfonline-com.ezproxy.lib.vutbr.cz/doi/full/10.1080/00140139.2013.877597?scroll=top&needAccess=true}
Rozsah míst kdě se dá applikovat Petrího sítě je velmí široký. Jako modelovací prostředek můžou sloužit například pro „reprezentací vzorů interakce“ mezí člověkem a autopilotém letadla(odkáz 1), nebo .... V nášem případě vysokourovňová Petrího síť modeluje a nasledně simuluje chování distribovaného sýstemu (termostat <-> teploměr) 

\subsection*{Znackovaná Petrího sít}
Prikladem pouziti znackovane petriho siti muze slouzit implementace "problemu Vecericich philosofu", coz je problem poprve predstaveny professorem Dijkstrou v roce 1965.
Je to obecne znamy problem paralelismu, ktery nazorne zobrazuje dva pripady chyb vznikajicich pri pararelnim vypoctu - vyhladoveni a uvaznuti.
% https://en.wikipedia.org/wiki/Dining_philosophers_problem#cite_note-formalization-2
% Petri nets theory and Modeling of Systems, p. 65-67, par 3.4.6
% TODO: naimplementovat a zobrazit
\subsection*{Vysokourovnova Petriho sit}
% TODO
\chapter{Implementace}
\subsection*{Knihovna SNAKES}
% https://www.ibisc.univ-evry.fr/~fpommereau/SNAKES/
SNAKES je knihovna v jazyce Python ktera nabizi vsechno potrebne pro definici a spousteni spousty typu Petriho siti. Cili knhovny SNAKES je nabidnout badatelum moznost rychle namodelovat novy napad. Zvlastni vlastnosti knihovny SNAKES je moznost vytvarzet Barevne Petriho site s pouzitim vyrazu v jazyce Python pro annotaci prechodu ci vstupnich nebo vystupnich sipek.

Zvlastni vlastnosti SNAKES je moznost implementace vlastnich rozsireni pro jednotlive casti Petriho siti, nebo pridani novych vlastnosti pro modelovani jinych podtypu Petriho siti. Tato vlastnost byla zvlast pouzita pro pridani rozsireni, podporujici vytvareni Petriho siti urcenych pro modelovani distribovanych systemu.
\subsection*{Implementace pluginu $mqtt_msg$}
Tento plugin ve svem zaklade umoznuje pouziti MQTT clientu pro provazani portu Petriho siti mezi sebou.
\subsection*{Vyuziti MQTT}
\subsection*{Planovac udalosti}
\chapter{Závěr}
\label{zaver}

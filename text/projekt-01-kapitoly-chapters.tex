%===============================================================================
% Autoři: Michal Bidlo, Bohuslav Křena, Jaroslav Dytrych, Petr Veigend a Adam Herout 2018
\chapter{Úvod}
\chapter{Prehled teorie}
\label{prehled}
V teto kapitole jsou popsany teoretické zaklady Petriho sítí. Taky tato kapitola se soustředí na popís několíka podtříd petrího site, zejmena na „vysokourovňové Petrího Sítě“ a „Časované Petrího Sítě“ a  včetně přehledu míst, kdě se Petrího sítě použivají, nebo hrají významnou roli.

           Petriho site
               V současné době se Petrího site často použivají jako modelovací prostředek pro modelovaní chovaní systemu, za účelem pochopení jeho slabších stránek. Použití Petrího síti pro tento učel je velice vhodne, malo modelovacích prostředku dokáže popsat ne jenom system jako celek, ale i zjistit slabší stránky jeho implementace.

Vezmeme například 

\chapter{Simulacni teorie}
\chapter{Simulace}
\chapter{Diskretni simulace}
\chapter{Simulace v realnem case}
\chapter{Planovac udalosti}
\chapter{Petriho site}
\chapter{Definice}
Petrího síť (PN) je pětice, PN = (P, T, I, O, M0) kde:
P = {p1, p2, ... , pn} je konečná množína míst, n > 0;
T = {t1, t2, ... , tk} je konečná množína přechodů, k > 0;
I = T -> P je vstupní funkce, propojující přechody s množínou vstupních míst;
O = P -> T je výstupní funkce, propojující výstupní místa s množínou přechodů;
M0 je množína počatečních znáček;
P ∧ T = ∅, P ∨ T ≠ ∅

Petrího síť může provest přechod 

\chapter{Znackovane petriho site}

\chapter{Casovane petriho site}
\chapter{Vysokourovnove petriho site}
\chapter{Applikace Petrího siti}

1. \url{https://www-tandfonline-com.ezproxy.lib.vutbr.cz/doi/full/10.1080/00140139.2013.877597?scroll=top&needAccess=true}
Rozsah míst kdě se dá applikovat Petrího sítě je velmí široký. Jako modelovací prostředek můžou sloužit například pro „reprezentací vzorů interakce“ mezí člověkem a autopilotém letadla(odkáz 1), nebo .... V nášem případě vysokourovňová Petrího síť modeluje a nasledně simuluje chování distribovaného sýstemu (termostat <-> teploměr) 
\chapter{Implementace}
\chapter{Knihovna SNAKES}


%===================================================================================
\chapter{Úvod}

Tento text slouží jako ukázkový obsah šablony a současně rekapituluje nejdůležitější informace z předpisů a poskytuje další užitečné informace, které budete potřebovat pro tvorbu technické zprávy ke svojí práci. Než se šablonou budete dále pracovat, je třeba vědět, jak ji správně použít. To je stručně uvedeno v~příloze \ref{jak}.

\chapter{Abstrakt}
\label{abstrakt}
\bigskip

\noindent Za prvé – na abstraktu záleží. Za druhé – není těžké ho napsat. Pojďme na to.

\subsection*{K čemu je abstrakt}

\subsection*{Kdy a jak psát Abstrakt}
\subsection*{Doporučená struktura abstraktu}

\paragraph{První část -- Jaký se řeší problém? Jaké je téma? Jaký je cíl textu?}
\begin{itemize}
  \item{Tato práce řeší.}
  \item{Cílem této práce je.}
  \item{Zaměřil jsem se na.}
\end{itemize}

\paragraph{Druhá část -- Jak je problém vyřešen? Cíl naplněn?}
\begin{itemize}
  \item{Zvolený problém jsem vyřešil pomocí toho a toho.}
  \item{V řešení bylo použito metody té, postupu toho a analýzy oné.}
  \item{Práce představuje algoritmus takový, který.}
  \item{Data jsem zpracovával pomocí těch a těchto nástrojů a provedl vyhodnocení takové.}
  \item{Podstatou našeho algoritmu je.}
\end{itemize}

\paragraph{Třetí část -- Jaké jsou konkrétní výsledky? Jak dobře je problém vyřešen?}
\begin{itemize}
  \item{Podařilo se dosáhnout úspěšnosti 87,3\,\%.}
  \item{V práci jsme vytvořili systém, který.}
  \item{Vytvořené řešení poskytuje ty a ty možnosti.}
  \item{Provedeným výzkumem jsme zjistili, že.}
\end{itemize}

\paragraph{Čtvrtá část -- Takže co? Čím je to užitečné Vědě a čtenáři?}
\begin{itemize}
  \item{Přínosem této práce je.}
  \item{Hlavním zjištěním je.}
  \item{Hlavním výsledkem je.}
  \item{Na základě zjištěných údajů je možné.}
  \item{Výsledky této práce umožňují.}
\end{itemize}


\paragraph{Nultá část -- O co jde? Kde jsme?}
\begin{itemize}
  \item{Práce je řešena v kontextu tom a tom.}
  \item{Nauka ta a ta se zabývá studiem toho a toho.}
  \item{Stavíme na těchto a oněch nedávných pokrocích v naší oblasti.}
\end{itemize}

\chapter{Příprava základní struktury práce} 
\label{struktura}


\chapter{Odevzdání práce} 
\label{odevzdani}

\chapter{Závěr}
\label{zaver}

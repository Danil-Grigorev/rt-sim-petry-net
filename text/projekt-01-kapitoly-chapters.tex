\chapter{Úvod}
\chapter{Prehled teorie}
\label{prehled}

V teto kapitole jsou popsany teoretické zaklady Petriho sítí. Taky tato kapitola se soustředí na popís několíka podtříd petrího site, zejmena na „vysokourovňové Petrího Sítě“ a „Časované Petrího Sítě“ a  včetně přehledu míst, kdě se Petrího sítě použivají, nebo hrají významnou roli. Taky bude zmineno o simulacni teorii, zamerene na simulaci Petriho siti v realnem case. Taky se popise pojem distribovanych systemu a ukaze se jejich vztah z Petriho siti.

\subsection{Petriho site}
V současné době se Petrího site často použivají jako modelovací prostředek pro modelovaní chovaní systemu, za účelem pochopení jeho slabších stránek. Použití Petrího síti pro tento učel je velice vhodne, malo modelovacích prostředku dokáže popsat ne jenom system jako celek, ale i zjistit slabší stránky jeho implementace.
\subsubsection{Definice}

Petrího síť ($PN$) se sestavuje ze ctrz prvku, $PN = \left(P, T, I, O\right)$ kde:
  \begin{itemize}
    \item $P = \left\{p_1, p_2, \cdots , p_n\right\}$ je konečná množína míst, $n >= 0$; \\
    \item $T = \left\{t_1, t_2, \cdots , t_m\right\}$ je konečná množína přechodů, $m >= 0$; \\
    \item $I = T \rightarrow P^\infty$ je vstupní funkce, propojující přechody s množínou vstupních míst; \\
    \item $O = P \rightarrow T^\infty$ je výstupní funkce, propojující výstupní místa s množínou přechodů; \\
  \end{itemize}
P $\vee$ T = $\varnothing$, P $\wedge$ T != $\varnothing$. 

\subsection{Graf Petriho site}

\subsubsection{Typy Petriho site}

\paragraph{Znackovane petriho site}
Znackovane petriho seti (ZPS) zavadi takovy pojem jako "znacka". "Znacka" je zakladni prvek v ZPS, umoznujici jeji vykonovani. Vykonovani se provadi provedenim prehodu, behem ktereho se vymazou znacky ze mnoziny vstupnich mist, a prida se odpovidajici pocet do vystupnich mist. Tato operace se muze vykonovat dokud nezbyde zadny povoleny prechod. Na tento myslence je zalozena simulace Petriho site. Prechod v PS je povolen, dokud kazde ze vstupnich mist propojenych s danym prechodem obsahuje alespon jednu znacku. 

Pri vykonovani seti Petri vznikaji dve posloupnosti - posloupnost znacek a posloupnost prechodu, ktere byly provedeny. Teto dve posloupnosti zcela popisuji vykonani Petriho seti.
%[Petri net theory and model... par:2.4 p 18-19]

Kazda ze sipek v Petriho siti ma vahu. Vaha reprezentuje cislo tokenu, ktere musi obsahovat vstupni misto pro to, aby prechod byl proveditelny.

Definice pak vypada takto:
Znackovaná Petriho sit je sestice $PN = \left(P, T, I, O, W, M_0\right)$
\begin{itemize}
  \item $P = \left\{p_1, p_2, \cdots , p_n\right\}$ je konečná množína míst, $n >= 0$; \\
  \item $T = \left\{t_1, t_2, \cdots , t_m\right\}$ je konečná množína přechodů, $m >= 0$; \\
  \item $I = T \rightarrow P^\infty$ je vstupní funkce, propojující přechody s množínou vstupních míst; \\
  \item $O = P \rightarrow T^\infty$ je výstupní funkce, propojující výstupní místa s množínou přechodů; \\
  \item $W = F \rightarrow \left\{1, 2, 3, \cdots \right\}$ je vahova funkce;
  \item $M_0 = TODO$
\end{itemize}
P $\vee$ T = $\varnothing$, P $\wedge$ T != $\varnothing$. TODO citace

Diky moznosti vykonovat Petriho sit, vznika prilezitost vytvorzit simulacni nastroj, ktery je zalozen na zminenych zasadech. Podrobny popis simulace viz implementace(odkaz).

\paragraph{Casovane petriho site}
Podtrida Casovanych petriho seti rozsiruje pojem petriho seti o jednu dulezitou zasadu. 
\paragraph{Vysokourovnove petriho site}
Vysokourovnove petriho site (High Level Petri Nets) jsou dalsim rozsirenim Petriho siti, ktere umoznuje jednodussi reprezentaci vypoctu, diky tomu ze sit ziskava nekolik dalsich vlastnosti. 

High Level Petri Net (HLPN) is a structure $HLPN = = \left( P; T; D; T_ype; P_re; P_{ost}; M_0\right) $ kde: 

\begin{itemize}
  \item $P$ je konecna mnozina mist. \\
  \item $T$ je konecna mnozina prechodu, pro kterou plati $\left(P \cup T = \varnothing\right)$. \\
  \item $D$ je konecna neprazna mnozina domen, kde kazdy element domeny je typ. \\ % is a non-empty finite set of non-empty domains where each element of D is called type.
  \item $T_{ype}$ : $P \cap T \rightarrow D$ je funkce pro pridani typu do mist a urceni typu prechodu.
  \item $P_{re}$, $P_{ost}$ : $TRANS$ $\rightarrow$ $\mu PLACE$ jsou pre a post kondicii pro ktere plati: \\
  \begin{center}
    \item $TRANS$ = $\left\{\left(t, m\right) | t \in T, m \in T_{ype} \left( t \right)\right\}$ \\
    \item $PLACE$ = $\left\{\left(p, g\right) | p \in P, g \in T_{ype} \left( p \right)\right\}$ \\
  \end{center}

  \item $M_0$ je $\mu PLACE$ je multimnozina pocatecnych znacek.
\end{itemize}
Citace pnstd-4.7.1.pdf
\paragraph*{Pravidlo pro provedeni prechodu}
Pokud je dane ze konecna multimnozina $T_\mu$ je zapnuta pro znacku $M$, tedy prechod v HLPN muze byt uskutecnen pro znacku $M'$ nasledujicim zpusobem:
\begin{center}
  $M' = M - Pre(T_\mu) + Post(T_\mu)$
\end{center}
TODO: citace pnstd p5.4
\subsection{HLPN Graf}
Graf HLPN ma nasledujici vlastnosti:
\begin{itemize}
  \item $Graf\:site$ je sestaveny s dvou prvku - $mist$ a $tansici$, ktere jsou propojeny mezi sebou sipkami($ark$).
  \item $Typy\:mist$ Kazde misto musi mit prirazeny jeden typ. Mnozina typu nesmi byt prazdna.
  \item $Znackovani\:mist$: je kolekce prvku, ktere museji odpovidat typu mista, ve kterem se nachazeji. Prvky se jmenuji $tokeny$ a muzou se opakovat.
  \item $Sipkove\:annotaci$: kazda sipka je annotovana vyrazem se sestavujicim z konstant, promenych ($x, y$) nebo funkci($f(x)$). Kazda promena je typovana, a vyrazy se provadeji prirazenim hodnot ke kazde ze promenych. Vyhodnoceni vyrazu produkuje colekci prvku prevzatych z vstupniho mista, ktera se muze opakovat.
  \item $Podminky\:pro\:prechod$: jsou booleove vyrazy ve tvaru $x < y$, ktere jsou v annotaci tranzice.
  \item $Deklaraci$: Vytvareji definici typu mist, funkci a promenych.
\end{itemize}

\subsection{Distribovany system}
Definice distribovaneho systemu zni:
Distribovany system je kolekce na sobe nezavislych pocitatcu, ktery konecny uzivatel vnima jako celek. 
% http://barbie.uta.edu/~jli/Resources/MapReduce&Hadoop/Distributed%20Systems%20Principles%20and%20Paradigms.pdf

Distribovane systemy museji pouzivat decentralizovane algoritmy, ktere pozaduji nasledujici: 1. Zadny prvek nesmi vedet informaci o celkovem stavu systemu. 2. Prvek uvazuje jenom na zaklade jeho stavu. 3. Chyba v zadnem z prvku ne ovlivni chovani algoritmu. 4. Neprovadi se synchronizace podle globalnich hodin. 

Z definice taky vypliva, ze distribovany system se obecne sestavuje s vice se opakujicich prvku, ktere funguji nezavisle na sobe a nemaji pristup ke sdilene pameti, neboli kazdy prvek ma svoi vlastni. Sdileni stavu ve pripade distribovaneho systemu se provadi pomoci zasilani zprav.

Tento koncept je velice pouzitelny pro modelovani. Prvky budou reprezentovany Petriho sitmi, provazane komunikacnimi kanaly. Jelikoz kazda petriho sit je prvek nezavisly na vnejsich podminkach, ale jen na soucasnem stavu mist a prechodu a taky jejich kombinaci, tak se da to prohlasit decentralizovanym algoritmem. Petriho sit si muzeme predstavit jako nezavisly blok, kdyzto pro kommunikaci mezi temito bloky se da pouzit protokol MQTT(odkaz), ktery diky svym vlastnostem dokaze obslouzit neomezene mnozstvi klientu najednou. Pripad implementace tepelneho rizeni(odkaz) nazorne ukazuje pouziti dane myslenky na praktice.
\subsection{Simulace}
\subsection{Diskretni simulace}
Diskretni simulace je zalozena ve svem zakladu na algoritmu znamem pod nazvem "NextEvent". Tento algoritmus se ridi nasledujicim pseudokodem:
\begin{algorithm}
  \caption{Diskretni simulace}\label{euclid}
  \begin{algorithmic}[1]
  \State $\text{nainicializuj simulaci, cas a planovac udalosti}$
  \While {dokud je naplanovana udalost}
  \State $\text{vyjmi prvni udalost ze seznamu}$
  \If {$cas_udalosti >= T_END$}
  \Return
  \EndIf
  \State $\text{nastav cas na cas udalosti}$
  \State $\text{proved popis chovani udalosti}$
  \EndWhile
  \end{algorithmic}
  \end{algorithm}
\subsection{Simulace v realnem case}
Simulace v realnem case se lisi of diskretni simulace v jedne vlastnosti. Beh takove simulace musi byt synchronizovan s relanim casem. Obecne posuv v casove rovine se provadi kvazidistantnim krokem, pricemz ten krok musi byt dostatecne maly, na to aby simulace stacila cist vstupni parametry. ??? % Real-Time Simulation of Electric Vehicles for Distribution System Operation Assessment, Characteristics of Real-Time Simulation

Simulace v realnem case upravena pro ucely tohoto projektu je zalozena na vyse zminenem algoritmu "Next Event" s jedinou zmenou, ktera pridava cekani na synchronizaci simulacniho casu s realnim nebo na prichod nove udalosti. Algoritmus v tom pripade vypada nasledujim zpusobem:

\begin{algorithm}
\caption{Real-time simulace}\label{euclid}
\begin{algorithmic}[1]
\State $\text{nainicializuj simulaci, cas a planovac udalosti}$
\While {dokud je naplanovana udalost}
\State $\text{podivej se na cas dalsi udalosti}$
\If {$cas_udalosti >= T_END$}
\Return
\EndIf
\State $\text{pockej na cas udalosti nebo na prichod nove udalosti}$
\If {nova udalost}:
    \State Continue
\EndIf
\State proved popis chovani udalosti
\EndWhile
\end{algorithmic}
\end{algorithm}

Tento algoritmus nachazi svoje pouziti ve mnoha simulacnich pripadech. Bezny priklad pouziti - pocitacove hry, kde uzivatele zaujme herni prostredi, ktere simuluje realni svet, vcetne casove roviny. Simulace v realnim case se taky pouziva, kdy obycejne diskretni simulace nestaci a cas neni zanedbatelny. Uzivatel muze chtit sledovat system v prubehu simulace, za ucelem zjisteni anomalii v chovani jeho prvku nebo zvyseni kvality systemu pred jeho nasazenim do provozu. Takovy pristup se pouziva v simulace "Hardware-in-the-loop"(odkaz).
\subsection{Hardware-in-the-loop(HWIL)}
\subsection{Applikace}
\paragraph{Obsah}

1. \url{https://www-tandfonline-com.ezproxy.lib.vutbr.cz/doi/full/10.1080/00140139.2013.877597?scroll=top&needAccess=true}
Rozsah míst kdě se dá applikovat Petrího sítě je velmí široký. Jako modelovací prostředek můžou sloužit například pro „reprezentací vzorů interakce“ mezí člověkem a autopilotém letadla(odkáz 1), nebo .... V nášem případě vysokourovňová Petrího síť modeluje a nasledně simuluje chování distribovaného sýstemu (termostat <-> teploměr) TODO

\subsection{Znackovaná Petrího sít}
Prikladem pouziti znackovane petriho siti muze slouzit implementace "problemu Vecericich philosofu", coz je problem poprve predstaveny professorem Dijkstrou v roce 1965.
Je to obecne znamy problem paralelismu, ktery nazorne zobrazuje dva pripady chyb vznikajicich pri pararelnim vypoctu - vyhladoveni a uvaznuti.
\url{https://en.wikipedia.org/wiki/Dining_philosophers_problem#cite_note-formalization-2} Petri nets theory and Modeling of Systems, p. 65-67, par 3.4.6
TODO: naimplementovat a zobrazit
\subsection{Vysokourovnova Petriho sit}
% TODO
\chapter{Implementace}
\subsection{Knihovna SNAKES}
\url{https://www.ibisc.univ-evry.fr/~fpommereau/SNAKES/}
SNAKES je knihovna v jazyce Python ktera nabizi vsechno potrebne pro definici a spousteni spousty typu Petriho siti. Cili knhovny SNAKES je nabidnout badatelum moznost rychle namodelovat novy napad. Zvlastni vlastnosti knihovny SNAKES je moznost vytvarzet Barevne Petriho site s pouzitim vyrazu v jazyce Python pro annotaci prechodu ci vstupnich nebo vystupnich sipek.

Zvlastni vlastnosti SNAKES je moznost implementace vlastnich rozsireni pro jednotlive casti Petriho siti, nebo pridani novych vlastnosti pro modelovani jinych podtypu Petriho siti. Tato vlastnost byla zvlast pouzita pro pridani rozsireni, podporujici vytvareni Petriho siti urcenych pro modelovani distribovanych systemu.
Tento plugin ve svem zaklade umoznuje pouziti MQTT clientu pro provazani portu Petriho siti mezi sebou.
\subsection{MQTT}
MQTT je protokol urceny pro komunikaci ve pripadech nizke propustnosti siti nebo jeji vysoke nespolehlivosti. V soucasne dobe svoje vyuziti nachazi v IoT (odkaz). MQTT zavadi takove pojmy jako klient a broker. Klient je tenka aplikace, schopna navazani spojeni s brokerem. Muze byt pouzite pro zarizeni, jejichz vykon je bod srazu. Broker je server, obsluhujici klienty, navazuje spojeni s klientem a provadi posilani zprav typu PUBLISH klientum, ktere sleduji tema odpovidajici teto zprave.

Kazda zprava posilana klientem se nepredava primo dalsimu klientovi, ale obsluhuje se brokerem. Zprava musi mit specifikovane tema, podle ktere broker urci, kterym klientum tu zpravu preposlat.

Zpravy museji navic mit specifikovany obsah, a QoS. QoS muze byt 3 typu:
\begin{itemize}
  \item $QoS_0$ Nanejvys jedno dodani \\
  \item $QoS_1$ Alespon jedno dodani \\
  \item $QoS_2$ Presne jedno dodani 
\end{itemize}

Pro navazani spojeni s brokerem po nastaveni sitoveho pripojeni klient musi nejdrive poslat zparavu $CONNECT$, po ktere server musi odpovedet zpravou typu $CONNACK$ podtvrzujici navazani spojeni. Potom klient je volen poslat zpravy typu $SUBSCRIBE$ s tematem na ktere chce dostavat zpravy. Server odpovida zpravou $SUBACK$. Potom uz zprava typu $PUBLISH$ z urcitym obsahem od klienta ktery provedl vsechny minule kroky bude zpracovana brokerem.

\subsection{Implementace pluginu}
Tento pododdil popisuje implementaci pluginu pro knihovny SNAKES, ktere zesnadnili nasledujici modelovani.

je urcen pro knihovny SNAKES a pridava podporu implementaci takovych rozsireni Petriho siti, jako Casovany prechod a Pravdepodobnostni prechod.

\subsubsection{Plugin $timed_pl$}
Toto rozsireni pouziva planovac udalosti pro spolehlive planovani spousteni tranzice do budoucna. Toto rozsireni pouziva jednoduchou logiku pro provedeni prechodu - ve chvili kdy prechod bude spustitelny, tak se zapne casovac a zkonzumuji odpovidajici tokeny ze vstupnich mist. Cekani se neresetuje pridanim dalsich tokenu do vstupnich mist. Ve pripade teto implementace prechod vezme noveprichozi tokeny ktere se objevi ve vstupnich mistech a tudiz zapnou tranzici. Po ukonceni cekani, tokeny budou preneseny do vystupnich mist, s nasledujcimi upravami, ktere muzou nastat podle definice HLPN(odkaz).

\subsubsection{Plugin $prob_pl$}
Toto rozsireni pridava moznost spojit nekolik tranzici do jedne skupiny, a tudiz rovnomerne rozdelit pravdepodobnost spousteni kazde z nich. Pro spojene prechody plati, ze museji mit stejnou sadu vstupnich mist. Pak se pri provedeni jedneho z nich se rozhodne na zaklade hodnoty jeho pravdepodobnosti jestli se tento prechod musi provest, nebo se provede nektery z jeho sousedu. Soucet pravdepodobnosti pro takovou skupinu prechodu musi byt vzdycky 1.

\subsubsection{Plugin $prior_pl$}
Tento plugin je urceny pro pridani priorit k pechodum v siti. Podle nastavene priority se behem simulace bude rozhodovat, ktere tranzice se vyhodnoti a provedou jako prvni. Priority muzou byt v rozsahu 0-100, a radi se sestupne. Pro nastaveni priority `Transition` ma argument `prior`.

\subsection{Planovac udalosti}
\url{https://docs.python.org/3/library/heapq.html}
Planovac udalosti je naimplementovan jako prioritni fronta za vyuziti vestavene knihovny heapq v Pythonu. Api naznacuje ze pro pridani a vyberu prvku se pouzivaji metody $heappush$ a $heappop$. \url{https://docs.python.org/3/library/heapq.html#theory} Priorita v takove fronte se urcuje ciselnou hodnotou, ktera se vklada do fronty. Tato hodnota muze byt prvnim prvkem v seznamu elementu. Druhym parametrem pro urceni priority je poradi, ve kterem prvek byl vlozen. Tato sada pravidel umoznuje vkladat zaznamy, obsahujici cas spusteni jako prvni prvek, a podle toho se zaradi na prisluhujici misto ve fronte. Metoda $heappop$ vybere prvek ze seznamu s nejvyssi prioritou, coz bude element s nejnizsi hodnotou casu.
\chapter{Aplikace}
\section{Tepelne rizeni}
Pro demonstraci praci simulatoru bylo rozhodnuto namodelovat dystribovany system modelujici chovani topeni, ktey nekteri z vas pravdepodobne pouzivaji u sebe doma. Takovy system se sestavuje z nekolika casti. Zakladnim prvkem je plynovy spotrebic, nebo karma. Ta ohriva vodu a uvadi ji do pohybu po potrubich. Pak kazdy z pokoju ma na starosti ?teplene cidlo? a rizeni, ktere udrzuje v pokoji nastavenou teplotu podle calendare. Kazdou hodinu se z tabulky teplot vezme aktualni teplota pro soucasnou hodiny a tento pokoj, pak rizeni rozhodne jestli ma zapnout topeni nebo ne. Pokud zadny z pokoju v soucasne dobe nema zapnute topeni, tak karma se musi vypnout za ucelem setreni plynu. Toto rizeni provadi vestaveny system, nebo je zcela mechanicky. Kazdy z tech prvku bude dale namodelovan Petriho sitmi.

Aby tento navrh co nejpresneji odpovidal realite, bylo rozhodnuto namodelovat chovani teploty v pokoji.
Nejdrive musime vedet jak vykonne je topeni v pokoji. Dale nas zajimaji tepelne ztraty a taky mnozstvi energii potrebne pro ohrati pokoje. Pro vypocet tohoto parametru pouzijeme nasledujici vzorecek. Zakladni parametrem rozhodujicim o tepelnych kapacite pokoje je plocha vnitrniho povrchu pokoje [V], respektive celkova plocha povrchu stopu, podlahy a sten. Pro ne je dulezte vedet koefficent ztrat tepla [K]. Zvlast se pocitaji okna a dvere, vzhledem k jejich vetshi teplovodivosti. Pak zbyva jenom zjistit rozdil teploty uvnitr pokoje a venku [delta T]. To vsechno dosadime do vzorecku Q[kW/h] = V*delta T*K/860 a zjistime celkove mnozstvi energii potrebne pro jeho ohrati. \cite{tep_calc}

Pro zjednoduseni vypoctu zanedbame parametry koefficentu ztrat tepla pro vsechny elementy, a vezmeme stredni hodnoty pro nekolik typickych druhu pokoju - kuchin, sklep a obyvaci pokoj. Kazdy z techto pokoju bude mit odlisny koefficent K a V. Pokoj bude mit 2 steny celici venku, kuchin bude mit 1. Pro sklep navic plati konstantni teplota vnejsiho prostredi, napriklad 5 stupnu. Taky se budeme pocitat s prijemnymi (skoro tropickymi) podminkami Ceske republiki - teplota venku je v prumeru v rozmezi 5 az 10 stupnu. \cite{cesko}
Pak koefficenty vypadaji nasledně:

\begin{table}[H]
	\vskip6pt
	\caption{Tabulka parametrů pokojů}
    \vskip6pt
	\centering
	\begin{tabular}{lllr}
		\toprule
		Typ pokoje & Koefficent tepelných ztrat & Objem $[m^3]$ & Očekavaný výkon topení $[W]$ \\
    \midrule
    Kuchin & 0.5 & $4*3*2.5$ & $1200$ \\
    Pokoj & 0.6 & $6*4*2.5$ & $2300$ \\
    Sklep & 0.8 & $3*3*2.2$ & $300$ \\
		\bottomrule
	\end{tabular}
	\label{tab:TepelneZtraty}
\end{table}

Na vypocet tepelnych ztrat pokoje jsem pouzil tento nastroj. \url{https://wpcalc.com/kalkulyator-teplopoter/}
Pro kuchin to vychazi zhruba nasledne:

\begin{table}[H]
	\vskip6pt
	\caption{Tabulka tepelných ztrát pokojů} 
    \vskip6pt
	\centering
	\begin{tabular}{llr}
		\toprule
		Typ pokoje & Ztráty $[kW/h]$ \\
		\midrule
		Kuchin & $0.7$ \\
    Pokoj & $1$ \\
    Sklep & $0.4$ \\
		\bottomrule
	\end{tabular}
	\label{tab:TepelneZtraty}
\end{table}

Cele chovani systemu odpovida zákonu zachovaní energii, a tak muzeme odvodit rovnici pro vypocet potrebneho jeji mnozstvi.

Modelovani chovani teploty venku je stochasticky process, ktery neni mozne spocitat presne. Proto byla navyrzena Petriho sit, vyuzivajici pravdepodobnostni prechody. Ma na vstupu rozsah teplot den - noc, a podle vybrane doby dne, za pouziti funkcii $cos$ modeluje zmenu aktualni teploty. Rozhodnuti pouziti funkce $cos$ vychazi z jednoducheho pozorovani prubehu teplot behem dne, a pereodicita zmeny teplot ji trochu odpovida, za predpokladu ze prubeh funkce zacina v nejteplejsi hodinu. %https://forecast.weather.gov/MapClick.php?lat=42.3758&lon=-71.1187&lg=english&FcstType=graphical
Pravdepodobnostni prechody modeluji nahodne jevy jako vyskit slunce nebo deste.


\chapter{Závěr}
\label{zaver}

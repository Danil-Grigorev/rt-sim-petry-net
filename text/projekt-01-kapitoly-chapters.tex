\chapter{Úvod}
\chapter{Přehled teorie}
\label{prehled}

V teto kapitole jsou popsaný teoretické základy Petřino sítí. Taky tato kapitola se soustředí na popis několika podtříd Petriho sítě, zejména na „vysokoúrovňové Petriho sítě“ a „Značkované Petriho Sítě“ a  včetně přehledu míst, kdě se tento modelovací prostředek používá, a hra významnou roli.

Dalším účelem této kapitoly je seznámení čtenáře s matematickými popisy Petriho síti a grafy Petriho sítí, které jsou nezbytné pro porozumění principu prací simulátoru.

Do obsahu patří popis protokolu MQTT \ref{subsec:mqtt-proto}, na kterém je založena externí komunikace simulátoru \ref{subsec:mqtt_impl}.

Taky se tady zmíní o simulační teorii, zaměřené na simulaci Petriho síti v reálném čase. Vyjasní se pojem distribuovaných systému \ref{subsec:distr_system}, HWIL \ref{subsec:hwil} a jejích vztah z Petriho sítěmi.

\subsection{Petriho sítě}
V současné době se Petriho sítě často používají jako modelovací prostředek pro modelovaní chovaní systému, za účelem pochopení jeho slabších stránek, ještě než se system nasadí do provozu. Vyplňuje, tím pádem, díru mezi slovním popisem, návrhem, takovými modelovacími prostředky jako \href{https://en.wikipedia.org/wiki/Unified_Modeling_Language}{UML} a skutečnou implementací nějakého projektu. Použití Petriho síti pro účel modelování je velice vhodné, jen velmi málo modelovacích prostředku dokáže popsat ne jenom system jako celek, ale i zjistit slabší stránky jeho implementace, ještě před implementační fází. Stejně tak platí i pro testovací fází -- některé chyby se dá objevit jenom po implementací, když to Petriho sítě částečně řeší tento problém ještě ve fází navrchu.

\paragraph{Definice}

Petriho síť ($PN$) se sestavuje ze čtyřce prvku, $PN = \left(P, T, I, O\right)$ kde:
  \begin{itemize}
    \item $P = \left\{p_1, p_2, \cdots , p_n\right\}$ je konečná množina míst, $n >= 0$; \\
    \item $T = \left\{t_1, t_2, \cdots , t_m\right\}$ je konečná množina přechodů, $m >= 0$; \\
    \item $I = T \rightarrow P^\infty$ je vstupní funkce, propojující přechody s množinou vstupních míst; \\
    \item $O = P \rightarrow T^\infty$ je výstupní funkce, propojující výstupní místa s množinou přechodů; \\
  \end{itemize}
P $\vee$ T = $\varnothing$, P $\wedge$ T != $\varnothing$.

\section{Typy Petriho sítě}

\subsection{Značkované Petriho sítě}
Značkované petriho seti (ZPS) zavadí takový pojem jako značka nebo \uv{token}. Značka je základní prvek v ZPS, umožnující její vykonávaní. Vykonávaní se provádí provedením přechodu, během kterého se vymaže odpovídající počet značek ze množiny vstupních míst, a do výstupních míst se potřebný počet značek vloží. Tato operace se muže vykonávat dokud nezbude žádný povoleny přechod. Na tento myšlence je založena simulace provádění Petriho sítě.

Přechod v PS je povolen, dokud každé ze vstupních míst propojených s daným přechodem obsahuje alespoň jednu značku.

Pri vykonávaní sítě Petri, vznikají dvě posloupnosti - posloupnost značek a posloupnost přechodu, které byly provedeny. Teto dve posloupnosti zcela popisuji vykonání Petriho sítě.
%[Petri net theory and model... par:2.4 p 18-19]

Kazda ze šipek v Petriho sítí ma váhu. Váha reprezentuje číslo tokenů, které se musí nacházet ve vstupním místě pro to, aby přechod byl proveditelný. Pří provedení váha určuje kolík tokenů se vyjme ze vstupního místa, nebo kolík se jích vloží do místa výstupního.

\paragraph{Definice}

Značkovaná Petriho sit je šestice $PN = \left(P, T, I, O, W, M_0\right)$
\begin{itemize}
  \item $P = \left\{p_1, p_2, \cdots , p_n\right\}$ je konečná množina míst, $n >= 0$; \\
  \item $T = \left\{t_1, t_2, \cdots , t_m\right\}$ je konečná množina přechodů, $m >= 0$; \\
  \item $I = T \rightarrow P^\infty$ je vstupní funkce, propojující přechody s množinou vstupních míst; \\
  \item $O = P \rightarrow T^\infty$ je výstupní funkce, propojující výstupní místa s množinou přechodů; \\
  \item $W = F \rightarrow \left\{1, 2, 3, \cdots \right\}$ je váhová funkce; \\
  \item $M_0$ je multimnožina počátečních značek;
\end{itemize}
P $\vee$ T = $\varnothing$, P $\wedge$ T != $\varnothing$.

Díky možnosti vykonávat Petriho sit, vzniká příležitost vytvořit simulační nastroj, který je založen na zmíněných teoretických základech. Podrobný popis simulace viz \ref{sec:implementation}.

\subsection{Časované petriho sítě}
Podtřída Časovaných petriho seti [Timed Petri Net] rozšiřuje pojem Značkovaných Petriho sítí o jednu důležitou zásadu -- přechod muže byt označen časovou funkcí. Aby se takový přechod provedl, ten čas musí uplynout, až potom se vyjmou tokeny ze vstupních míst, a vloží se do výstupních. Časová funkce nemusí být konstantní, ale v rámci této prací se použilo jenom konstantní čekaní na přechod, jelikož chovaní simulátoru musí být deterministické. Více o implementačních detailech viz. \ref{subsec:timed_pl}.

\subsection{Vysokoúrovňové Petriho sítě}
Vysokoúrovňové petriho sítě (High Level Petri Nets) jsou dalším rozšířením Petriho sítí, které umožňují jednodušší modelovaní takových procesů, pro které jsou důležité výpočty a matematické výrazy. Celkově, je tento druh Petriho sítě pohodlnější pro modelovaní toku řízení kódu nějakého programu, a ve své podstatě připomíná grafické programování.

\paragraph{Definice}

High Level Petri Net (HLPN) je struktura $HLPN = \left( P; T; D; T_{ype}; P_{re}; P_{ost}; M_0\right) $ kde \cite[p.11--12]{pnstd54}:

\begin{itemize}
  \item $P$ je konečná množina míst. \\
  \item $T$ je konečná množina přechodu, pro kterou platí $\left(P \cup T = \varnothing\right)$. \\
  \item $D$ je konečná neprázdná množina domén, kde kazdy element domény se jmenuje typ. \\
  \item $T_{ype}$ : $P \cap T \rightarrow D$ je funkce pro přidaní typu do míst a určeni typu přechodu.
  \item $P_{re}$, $P_{ost}$ : $TRANS$ $\rightarrow$ $\mu PLACE$ jsou pre a post kondici pro které platí: \\
  \begin{center}
    \item $TRANS$ = $\left\{\left(t, m\right) | t \in T, m \in T_{ype} \left( t \right)\right\}$ \\
    \item $PLACE$ = $\left\{\left(p, g\right) | p \in P, g \in T_{ype} \left( p \right)\right\}$ \\
  \end{center}
  \item $M_0$ je $\mu PLACE$ je multimnožina počátečních značek.
\end{itemize}

\subsection{Pravidlo pro provedeni přechodu}
Pokud je dane ze konečná multimnožina $T_\mu$ je zapnuta pro značku $M$, tedy přechod v $HLPN$ muže byt proveden pro značku $M'$ následujícím způsobem \cite[p.~12]{pnstd54}:
\begin{center}
  $M' = M - Pre(T_\mu) + Post(T_\mu)$
\end{center}


\subsection{$HLPN$ Graf}
Graf $HLPN$ ma následující vlastnosti:
\begin{itemize}
  \item $Graf\:site$ je sestaveny s dvou prvku - $mist$ a $tansici$, které jsou propojeny mezi sebou šipkami ($arks$).
  \item $Typy\:mist$ Každé místo musí mít přiraženy jeden typ. Množina typu nesmí byt prázdna.
  \item $Znackovani\:mist$: je kolekce prvku, které musejí odpovídat typu místa, ve kterém se nacházejí. Prvky se jmenuji $tokeny$ a můžou se opakovat.
  \item $Sipkove\:annotaci$: kazda šipka je anotovaná výrazem se sestavujícím z konstant, proměnách ($x, y$) nebo funkci($f(x)$). Kazda proměna je typovaná, a výrazy se provádějí přiražením hodnot ke každé ze proměnách. Vyhodnoceni výrazu produkuje kolekci prvku převzatých ze vstupního místa. Kolekce se muže opakovat.
  \item $Podminky\:pro\:přechod$: jsou booleové výrazy ve tvaru $x < y$, které jsou zapsané v anotaci tranzite.
  \item $Deklaraci$: Vytvářejí definici typu míst, funkci a promenych.
\end{itemize}

\subsection{Aplikace Petriho sítí}

\subsection{Značkovaná Petriho sít}
Rozsah míst kdě se dá applikovat Petriho sítě je velmí široký. Jako modelovací prostředek můžou sloužit například pro reprezentací vzorů interakce a vznikájicich konfliktů mezí člověkem a autopilotém letadla, jak je popsané v nasledujícím \href{https://www-tandfonline-com.ezproxy.lib.vutbr.cz/doi/full/10.1080/00140139.2013.877597}{článku}.

Příkladem použiti značkované petriho sítí muže taky sloužit implementace \uv{Dining philosophers problem} \cite[p.65--67]{PNandMoS}, coz je problém poprvé představeny profesorem Dijkstrou v roce 1965.
Je to obecné známy problém paralelismu, který názorné zobrazuje dva případy chyb vznikajících pri paralelním vypočtu - vyhladověni a uváznutí \cite{dining_philosophers}.

\subsection{Vysokoúrovňová Petriho sit}

V našem případě vysokoúrovňová Petriho síť modeluje a nasledně simuluje chování systému topení.

\subsection{Distribuovaný system}
\label{subsec:distr_system}

\paragraph{Definice}

Definice distribuovaného systému zni:
Distribuovaný system je kolekce na sobe nezavislych pocitatcu, ktery konecny uzivatel vnima jako celek.
% http://barbie.uta.edu/~jli/Resources/MapReduce&Hadoop/Distributed%20Systems%20Principles%20and%20Paradigms.pdf

Distribuované systémy musejí používat decentralizované algoritmy, které požaduji následující:
\begin{enumerate}
  \item Žádný prvek nesmí vědět informaci o celkovém stavu systému.
  \item Prvek uvazuje jenom na základe jeho stavu.
  \item Chyba v žádném z prvku ne ovlivni chovaní algoritmu.
  \item Neprovádí se synchronizace podle globálních hodin.
\end{enumerate}

Z definice taky vypliva, ze distribuovany system se obecne sestavuje s vice se opakujicich prvku, ktere funguji nezavisle na sobe a nemaji pristup ke sdilene pameti, neboli kazdy prvek ma svoi vlastni. Sdileni stavu ve pripade distribovaneho systému se provadi pomoci zasilani zprav.

Tento koncept je velice pouzitelny pro modelovani. Prvky budou reprezentovany Petriho siťěmi, provazane komunikacnimi kanaly. Jelikoz kazda petriho sit je prvek nezavisly na vnejsich podminkach, ale jen na soucasnem stavu mist a přechodu a taky jejich kombinaci, tak se da to prohlasit decentralizovanym algoritmem. Petriho sit si muzeme predstavit jako nezavisly blok, kdyzto pro kommunikaci mezi temito bloky se da pouzit protokol \href{http://docs.oasis-open.org/mqtt/mqtt/v3.1.1/csprd02/mqtt-v3.1.1-csprd02.html}{MQTT}, ktery diky svym vlastnostem dokaze obslouzit neomezene mnozstvi klientu najednou. Pripad implementace tepelneho rizeni(odkaz) nazorne ukazuje pouziti dane myslenky na praktice.
\subsection{Simulace}
\subsection{Diskretni simulace}
Diskrétní simulace je založena ve svém základu na algoritmu známém pod názvem \uv{NextEvent}. Tento algoritmus se řídí nasledujícím pseudokodem:
\begin{algorithm}
  \caption{Diskretni simulace}\label{euclid}
  \begin{algorithmic}[1]
  \State $\text{nainicializuj simulaci, cas a planovac udalosti}$
  \While {dokud je naplanovana udalost}
  \State $\text{vyjmi prvni udalost ze seznamu}$
  \If {$cas_udalosti >= T_END$}
  \Return
  \EndIf
  \State $\text{nastav cas na cas udalosti}$
  \State $\text{proved popis chovani udalosti}$
  \EndWhile
  \end{algorithmic}
  \end{algorithm}
\subsection{Simulace v realnem case}
Simulace v realnem case se lisi of diskretni simulace v jedne vlastnosti. Beh takove simulace musi byt synchronizovan s relanim casem. Obecne posuv v casove rovine se provadi kvazidistantnim krokem, pricemz ten krok musi byt dostatecne maly, na to aby simulace stacila cist vstupni parametry. ??? % Real-Time Simulation of Electric Vehicles for Distribution System Operation Assessment, Characteristics of Real-Time Simulation

Simulace v realnem case upravena pro ucely tohoto projektu je zalozena na vyse zminenem algoritmu \uv{Next Event} s jedinou zmenou, ktera pridava cekani na synchronizaci simulacniho casu s realnim nebo na prichod nove udalosti. Algoritmus v tom pripade vypadá nasledujim zpusobem:

\begin{algorithm}
\caption{Real-time simulace}\label{euclid}
\begin{algorithmic}[1]
\State $\text{nainicializuj simulaci, cas a planovac udalosti}$
\While {dokud je naplanovana udalost}
\State $\text{podivej se na cas dalsi udalosti}$
\If {$cas_udalosti >= T_END$}
\Return
\EndIf
\State $\text{pockej na cas udalosti nebo na prichod nove udalosti}$
\If {nova udalost}:
    \State Continue
\EndIf
\State proved popis chovani udalosti
\EndWhile
\end{algorithmic}
\end{algorithm}

Tento algoritmus nachazi svoje pouziti ve mnoha simulacnich pripadech. Bezny priklad pouziti - pocitacove hry, kde uzivatele zaujme herni prostredi, ktere simuluje realni svet, vcetne casove roviny. Simulace v realnim case se taky pouziva, kdy obycejne diskretni simulace nestaci a cas neni zanedbatelny. Uzivatel muze chtit sledovat system v prubehu simulace, za ucelem zjisteni anomalii v chovani jeho prvku nebo zvyseni kvality systému pred jeho nasazenim do provozu. Takovy pristup se pouziva v simulace \hyperref[subsec:hwil]{\uv{Hardware-in-the-loop}}.

\subsection{Hardware-in-the-loop(HWIL)}
\label{subsec:hwil}

Hardware-in-the-loop nebo $HIL$ je technika použivaná pro vývoj a testovaní řídicích časti komplexních systemů. Diky níž se fizické časti systemu nahrazují jejích simulaci a umožnuji plynulé testovaní ještě před nasazenim do provozu \cite{hil}.

\subsection{MQTT protokol}
\label{subsec:mqtt-proto}

\href{http://mqtt.org/}{MQTT} je protokol urceny pro komunikaci ve pripadech nizke propustnosti siti nebo jeji vysoke nespolehlivosti. V soucasne dobe svoje vyuziti nachazi v \href{https://en.wikipedia.org/wiki/Internet_of_things}{IoT}. MQTT zavadi takove pojmy jako \href{par:client}{klient}, \href{par:broker}{broker}, \href{par:message}{zpráva} a \href{par:topic}{téma}. Dalé nasledují popisy důležitých časti protokolu MQTT z daneho zdroje \cite{mqtt}.

\paragraph{MQTT Klient}
\label{par:client}

Klient je tenka aplikace, schopna navazani spojeni s brokerem. K jeho vlastnostém patří podpora odebíraní zpráv pomoci filtrovaní na zakladě parametru, zvanemu \uv{topic} \ref{par:topic}. MQTT klient muže byt použit pro zarizeni, jejichz vykon je bod srazu, díky lehkosti MQTT protokolu.

\paragraph{MQTT Broker}
\label{par:broker}

Broker je server, obsluhujici klienty, navazuje spojeni s klientem a provadi posilani zprav typu $PUBLISH$ klientům, které odebírají téma, nachazející v hlavičce této zpravy.

\paragraph{Téma}
\label{par:topic}

Kazda zprava posilana klientem se nepredava primo dalsimu klientovi, ale obsluhuje se brokerem. Zprava musi mit specifikovane tema, podle ktere broker urci, kterym klientům tu zpravu musí přeposlat. 

\paragraph{Zpráva}
\label{par:message}

Zpravy museji navic mit specifikovany obsah, a QoS. QoS muze byt 3 typu:
\begin{itemize}
  \item $QoS_0$ -- Nanejvys jedno dodani
  \item $QoS_1$ -- Alespon jedno dodani
  \item $QoS_2$ -- Presne jedno dodani
\end{itemize}

\paragraph{Tok nastavení}

Pro navazani spojeni s brokerem po nastaveni sitoveho pripojeni klient musi nejdrive poslat zparavu $CONNECT$, po ktere server musi odpovedet zpravou typu $CONNACK$ podtvrzujici navazani spojeni. Potom klient je volen poslat zpravy typu $SUBSCRIBE$ s tematem na ktere chce dostavat zpravy. Server odpovida zpravou $SUBACK$. Potom uz zprava typu $PUBLISH$ z urcitym obsahem od klienta ktery provedl vsechny minule kroky bude zpracovana brokerem.


\chapter{Implementace}
\label{sec:implementation}

\subsection{Knihovna $SNAKES$}

\href{https://www.ibisc.univ-evry.fr/~fpommereau/SNAKES/}{$SNAKES$} -- je knihovna v jazyce Python ktera nabizi vsechno potrebne pro definici a spousteni několíka variant Petriho sítí. Cilem knhovny $SNAKES$ je nabidnout badatelum moznost rychle namodelovat novy napad. Zvlastni vlastnosti knihovny $SNAKES$ je moznost vytvarzet Barevne Petriho site s pouzitim vyrazů v jazyce Python pro annotaci přechodu ci vstupnich nebo vystupnich šipek. \cite{snakes}

Zvlastni vlastnosti $SNAKES$ je moznost implementace vlastnich rozsireni pro jednotlive casti Petriho sítí, nebo pridani novych vlastnosti pro modelovani jinych podtypu Petriho sítí. Tato vlastnost byla zvlast pouzita v této prací pro pridani rozsireni, podporujici vytvareni Petriho sítí urcenych pro modelovani distribuovanych systému.
Jednou z nejdůležitějších pluginu je plugin umoznující napojení \hyperref[sec:aplikace-mqtt]{MQTT clientů} pro provazani portu Petriho sítí mezi sebou.

\subsection{Implementace MQTT}
\label{subsec:mqtt_impl}
Pro implementací mqtt klientu v simulatoru byla použitá knihovna \href{https://pypi.org/project/paho-mqtt/}{\texttt{paho-mqtt}}. Tato knihovna obsahuje všechno potřebné pro nastavení komunikačních kanálů mezí porty vzdalených Petriho sítí. Na základě použití této knihovny konečný uživatel může nastavit zvolené místo v Petriho sítí jako vstupní nebo vystupní port. viz \ref{code:remote-in-out}. Jejích vzhled je nasledující:
\begin{figure}[hbt]
  \centering
  \includegraphics[width=0.5\textwidth]{obrazky-figures/port-in.png}
  \caption{Příklad vstupního portu}
  \label{port-in}
\end{figure}

\begin{figure}[hbt]
  \centering
  \includegraphics[width=0.5\textwidth]{obrazky-figures/port-out.png}
  \caption{Příklad výstupního portu}
  \label{port-out}
\end{figure}

Jak si můžete povšimnout, na rozdíl od ostatních míst sítě, vstupní a výstupní porty se označují barvou a dodatečným popisem. Pro vstupní port \ref{port-in} je to \texttt{Listening on: NET/TOPIC} a zelení barva, \texttt{Sending to: NET/TOPIC} pro výstupní \ref{port-out}. Ten, ve své řáde je označen červeně.

Je předem definované, že každá simulační instance odebírá zprávy z tématu \texttt{control}. Je to ekvivalent broadcastové adresy u TCP/IP zásobníku.

Implementace zevnítř vypadá podobně reprezentací samotných portů. Sít, pří volaní metody \texttt{add\_remote\_input} nastaví týp místa na vstupní, pak pošle zprávu na téma \texttt{control} obsahující dvojicí [jméno sítě, jméno místa], a uloží zprávu do čekací fronty. Pak, když se k brokeru přípojí nějaká simulační instance s požadovanou dvojící síť--místo, tak se to oznamí všem ostatním simulačním instancím. To se zase provede zasilaním zprávy na téma \texttt{control}. Hned po příjetí takové zprávy se port nastaví a bude vykonavat svou čínnost.

Co se tyče směřovaní zpráv, tak na jeden vstupní port může být přípojeno 0 až několík vystupních portů, nebo jinymí slovy vstupní port příjma všechny zprávy na svoje temá.

\subsection{Implementace pluginů}
\label{sec:plug-impl}
Tento pododdil popisuje implementaci růyných pluginu pro knihovnu $SNAKES$, které zesnadnuji návrch Petriho sítí a její modelovaní. Každý z níže popsaných pluginu se da  použít v kodu po jejích importu podle tohoto návodu \todo{Add $SNAKES$ plugin include page}

\subsubsection{Plugin $timed_pl$}
\label{subsec:timed_pl}
Toto rozsireni pouziva planovac udalosti pro spolehlive planovani spousteni tranzice do budoucna. Toto rozsireni pouziva jednoduchou logiku pro provedeni přechodu - ve chvili kdy přechod bude spustitelny, tak se zapne casovac a zkonzumuji odpovidajici tokeny ze vstupnich mist. Cekani se neresetuje pridanim dalsich tokenu do vstupnich mist. Ve pripade teto implementace přechod vezme noveprichozi tokeny ktere se objevi ve vstupnich mistech a tudiz zapnou tranzici. Po ukonceni cekani, tokeny budou preneseny do vystupnich mist, s nasledujcimi upravami, ktere můžou nastat podle definice HLPN(odkaz).
\begin{figure}[hbt]
  \centering
  \includegraphics[width=0.3\textwidth]{obrazky-figures/timed-transition.png}
  \caption{Příklad časovaného přechodu}
  \label{timed-transition}
\end{figure}

\subsubsection{Plugin $prob_pl$}
\label{subsec:prob_pl}
Toto rozsireni pridava moznost spojit nekolik tranzici do jedne skupiny, a tudiz rovnomerne rozdelit pravdepodobnost spousteni kazde z nich. Pro spojene přechody plati, ze museji mit stejnou sadu vstupnich mist. Pak se pri provedeni jedneho z nich se rozhodne na zaklade hodnoty jeho pravdepodobnosti jestli se tento přechod musi provest, nebo se provede nektery z jeho sousedu. Soucet pravdepodobnosti pro takovou skupinu přechodu musi byt vzdycky 1.
\begin{figure}[hbt]
  \centering
  \includegraphics[width=0.6\textwidth]{obrazky-figures/prob-transition.png}
  \caption{Příklad pravděpodobnostního přechodu}
  \label{prob-transition}
\end{figure}

\subsubsection{Plugin $prior_pl$}
\label{subsec:prior_pl}
Tento plugin je urceny pro pridani priorit k pechodum v siti. Podle nastavene priority se behem simulace bude rozhodovat, ktere tranzice se vyhodnoti a provedou jako prvni. Priority můžou byt v rozsahu 0 -- 100, a radi se sestupne. Pro nastaveni priority element `Transition` má dodatečný a nepovinný parametr `prior`. 
\begin{figure}[hbt]
  \centering
  \includegraphics[width=0.3\textwidth]{obrazky-figures/prior-transition.png}
  \caption{Příklad prioritního přechodu s prioritou $1$}
  \label{prior-transition}
\end{figure}

\subsection{Plugin $sim_pl$}
\label{sec:aplikace-mqtt}
Na implementací tohoto pluginu je založená práce ostatních pluginů ze skupiny \ref{sec:plug-impl}. Je ve své podstatě spojovacím prvkém mezi simulatorem, $MQTT$ klientem a sětí Petri. Obsahuje přetežovaní metod ze knihovny $SNAKES$ pro vykreslování prvků, které jsou rozšiřené pluginy. Taky přidavá zasadní metody pro nastavení Petriho sítě, umožnující použítí simulatoru a provadění spouštění sítě \ref{code:add_simulator}.
\newcounter{codecounter}
\begin{lstlisting}[language=Python, escapeinside={(*}{*)}, numbers=right]
  sim = PNSim()
  n = PetriNet('Sample net')
  n.add_simulator(sim) (*\label{code:add_simulator}*)
\end{lstlisting}

Stejně tak podporuje nastavení vzdalených portů pro předaní tokenů jiným sítím připojeným na stejnou adresu $MQTT$ brokeru \ref{code:remote_in}, \ref{code:remote_out}.

\begin{lstlisting}[language=Python, escapeinside={(*}{*)}, numbers=right, label={code:remote-in-out}]
  n.add_remote_input(it, 'temp_gen/Measurement') (*\label{code:remote_in}*)
  n.add_remote_output(hen, 'boiler_logic/Sensory_input') (*\label{code:remote_out}*)
\end{lstlisting}


\subsection{Planovac udalosti}
\url{https://docs.python.org/3/library/heapq.html}
Planovac udalosti je naimplementovan jako prioritni fronta za vyuziti vestavene knihovny heapq v Pythonu. Api naznacuje ze pro pridani a vyberu prvku se pouzivaji metody $heappush$ a $heappop$. \url{https://docs.python.org/3/library/heapq.html#theory} Priorita v takove fronte se urcuje ciselnou hodnotou, ktera se vklada do fronty. Tato hodnota muze byt prvnim prvkem v seznamu elementu. Druhym parametrem pro urceni priority je poradi, ve kterem prvek byl vlozen. Tato sada pravidel umoznuje vkladat zaznamy, obsahujici cas spusteni jako prvni prvek, a podle toho se zaradi na prisluhujici misto ve fronte. Metoda $heappop$ vybere prvek ze seznamu s nejvyssi prioritou, coz bude element s nejnizsi hodnotou casu.
\chapter{Aplikace}
\section{Tepelne rizeni}
Pro demonstraci praci simulatoru bylo rozhodnuto namodelovat dystribovany system modelujici chovani topení, ktey nekteri z vas pravdepodobne pouzivaji u sebe doma. Takovy system se sestavuje z nekolika casti. Zakladnim prvkem je plynovy spotrebic, nebo karma. Ta ohriva vodu a uvadi ji do pohybu po potrubich. Pak kazdy z pokoju ma na starosti ?teplene cidlo? a rizeni, ktere udrzuje v pokoji nastavenou teplotu podle calendare. Kazdou hodinu se z tabulky teplot vezme aktualni teplota pro soucasnou hodiny a tento pokoj, pak rizeni rozhodne jestli ma zapnout topení nebo ne. Pokud zadny z pokoju v soucasne dobe nema zapnute topení, tak karma se musi vypnout za ucelem setreni plynu. Toto rizeni provadi vestaveny system, nebo je zcela mechanicky. Kazdy z tech prvku bude dale namodelovan Petriho siťěmi.

Aby tento navrh co nejpresneji odpovidal realite, bylo rozhodnuto namodelovat chovani teploty v pokoji.
Nejdrive musime vedet jak vykonne je topení v pokoji. Běžné hodnoty pro tři druhy pokujů mužete videt v nasledujici \hyperref[tab:TepelneZtraty]{tabulce}. Dale nas zajimaji tepelné ztraty a taky mnozstvi energii potrebne pro ohrati pokoje. Pro vypocet tohoto parametru pouzijeme nasledující vzorecek. Zakladni parametrem rozhodujicim o tepelnych kapacite pokoje je plocha vnitrniho povrchu pokoje $V$, respektive celkova plocha povrchu stopu, podlahy a sten. Pro ně je dulezte vědět koefficent ztrat tepla $K$. Zvlast se pocitaji okna a dvere, vzhledem k jejich vetshi teplovodivosti. Pak zbyva jenom zjistit rozdil současné teploty uvnitr pokoje a očekavane teploty pro tuto hodínu - $\Delta{T}$. To vsechno dosadime do vzorecku a zjistime celkové mnozstvi energii potrebne pro jeho ohrati. \cite{tep_calc}
\begin{center}
  $Q[kW/h] = \frac{V*\Delta{T}*K}{860}$
\end{center}

Pro zjednoduseni vypoctu zanedbame parametry koefficentu ztrat tepla pro vsechny elementy, a vezmeme stredni hodnoty pro nekolik typickych druhu pokoju - kuchin, sklep a obyvaci pokoj. Kazdy z techto pokoju bude mit odlisny koefficent K a V. Pokoj bude mit 2 steny celici venku, kuchin bude mit 1. Pro sklep navic plati konstantni teplota vnejsiho prostredi, napriklad 5 stupnu. Taky se budeme pocitat s prijemnymi (skoro tropickymi) podminkami Ceske republiki - teplota venku je v prumeru v rozmezi 5 az 10 stupnu. \cite{cesko}
Pak koefficenty vypadáji nasledně:

\begin{table}[H]
	\vskip6pt
	\caption{Tabulka parametrů pokojů}
    \vskip6pt
	\centering
	\begin{tabular}{lllr}
		\toprule
		Typ pokoje & Koefficent tepelných ztrat & Objem $[m^3]$ & Očekavaný výkon topení $[W]$ \\
    \midrule
    Kuchin & 0.5 & $4*3*2.5$ & $1200$ \\
    Pokoj & 0.6 & $6*4*2.5$ & $2300$ \\
    Sklep & 0.8 & $3*3*2.2$ & $300$ \\
		\bottomrule
	\end{tabular}
	\label{tab:Parametry}
\end{table}

Na vypocet tepelnych ztrat pokoje jsem pouzil tento nastroj. \url{https://wpcalc.com/kalkulyator-teplopoter/}
Vychazí to zhruba nasledne:

\begin{table}[H]
	\vskip6pt
	\caption{Tabulka tepelných ztrát pokojů} 
    \vskip6pt
	\centering
	\begin{tabular}{llr}
		\toprule
		Typ pokoje & Ztráty $[kW/h]$ \\
		\midrule
		Kuchin & $0.7$ \\
    Pokoj & $1$ \\
    Sklep & $0.4$ \\
		\bottomrule
	\end{tabular}
	\label{tab:TepelneZtraty}
\end{table}

Cele chovani systému odpovida zákonu zachovaní energii, a tak muzeme odvodit rovnici pro vypocet potrebneho jeji mnozstvi.


Modelovani chovani teploty venku je stochasticky process, ktery neni mozne spocitat presne. Proto byla navyrzena Petriho sit, vyuzivajici pravděpodobnostní přechody. Ma na vstupu rozsah teplot den - noc, a podle vybrane doby dne, za pouziti funkcii $cos$ modeluje zmenu aktualni teploty. Rozhodnuti pouziti funkce $cos$ vychazi z jednoducheho pozorovani prubehu teplot behem dne, a pereodicita zmeny teplot ji trochu odpovida, za predpokladu ze prubeh funkce zacina v nejteplejsi hodinu. %https://forecast.weather.gov/MapClick.php?lat=42.3758&lon=-71.1187&lg=english&FcstType=graphical
Pravděpodobnostní přechody modeluji nahodne jevy jako oteplení nebo ochlazení kvůli východu sluncie nebo jiné.


\chapter{Závěr}
\label{zaver}
